%%%%%%%%%%%%%%%%%%%%%%%%%%%%%%%%%%%%%%%%%%%%%%%%%%%%%%%%%%%%%%%%%%%%%%%%%%%%%%%%
%2345678901234567890123456789012345678901234567890123456789012345678901234567890
%        1         2         3         4         5         6         7         8
%
% Slightly modified by Z. Zhang for FG 2018
%

%\documentclass[letterpaper, 10 pt, conference]{ieeeconf}  % Comment this line out
                                                          % if you need a4paper
\documentclass[a4paper, 10pt, conference]{ieeeconf}      % Use this line for a4
                                                          % paper
\usepackage{FG2018}

%\FGfinalcopy % *** Uncomment this line for the final submission



\IEEEoverridecommandlockouts                              % This command is only
                                                          % needed if you want to
                                                          % use the \thanks command
\overrideIEEEmargins
% See the \addtolength command later in the file to balance the column lengths
% on the last page of the document

% The following packages can be found on http:\\www.ctan.org
%\usepackage{graphics} % for pdf, bitmapped graphics files
%\usepackage{epsfig} % for postscript graphics files
%\usepackage{mathptmx} % assumes new font selection scheme installed
%\usepackage{times} % assumes new font selection scheme installed
%\usepackage{amsmath} % assumes amsmath package installed
%\usepackage{amssymb}  % assumes amsmath package installed

\def\FGPaperID{****} % *** Enter the FG 2018 Paper ID here

\title{\LARGE \bf
Preparation of Papers for IEEE Sponsored Conferences \& Symposia
}

%use this in case of a single affiliation
%\author{\parbox{16cm}{\centering
%    {\large Huibert Kwakernaak}\\
%    {\normalsize
%    Faculty of Electrical Engineering, Mathematics and Computer Science, University of Twente, Enschede, The Netherlands\\}}
%    \thanks{This work was not supported by any organization.}% <-this % stops a space
%}

%use this in case of several affiliations
\author{\parbox{16cm}{\centering
    {\large Huibert Kwakernaak$^1$ and Pradeep Misra$^2$}\\
    {\normalsize
    $^1$ Faculty of Electrical Engineering, Mathematics and Computer Science, University of Twente, Enschede, The Netherlands\\
    $^2$ Department of Electrical Engineering, Wright State University, Dayton, USA}}
    \thanks{This work was not supported by any organization}% <-this % stops a space
}

\begin{document}

\ifFGfinal
\thispagestyle{empty}
\pagestyle{empty}
\else
\author{Anonymous FG 2018 submission\\ Paper ID \FGPaperID \\}
\pagestyle{plain}
\fi
\maketitle



%%%%%%%%%%%%%%%%%%%%%%%%%%%%%%%%%%%%%%%%%%%%%%%%%%%%%%%%%%%%%%%%%%%%%%%%%%%%%%%%
\begin{abstract}

Abstract. 

\end{abstract}


%%%%%%%%%%%%%%%%%%%%%%%%%%%%%%%%%%%%%%%%%%%%%%%%%%%%%%%%%%%%%%%%%%%%%%%%%%%%%%%%
\section{Introduction}

Intro

\subsection{XXX}

XXX

\begin{table}
\caption{An Example of a Table}
\label{table_example}
\begin{center}
\begin{tabular}{|c||c|}
\hline
One & Two\\
\hline
Three & Four\\
\hline
\end{tabular}
\end{center}
\end{table}



\subsection{Numbering}

Number reference citations consecutively in square brackets \cite{c1}.
 The sentence punctuation follows the brackets \cite{c2}.
 Refer simply to the reference number, as in \cite{c3}.
 Do not use ``ref. \cite{c3}'' or ``reference \cite{c3}''.
Number footnotes separately in superscripts\footnote{This is a footnote}
Place the actual footnote at the bottom of the column in which it is cited.
Do not put footnotes in the reference list.
Use letters for table footnotes (see Table I).

\subsection{Equations}

Number equations consecutively with equation numbers in parentheses flush
 with the right margin, as in (1). To make your equations more compact
 you may use the solidus (/), the exp. function, or appropriate exponents.
  Italicize Roman symbols for quantities and variables, but not Greek symbols.
   Use a long dash rather then hyphen for a minus sign. Use parentheses to avoid
    ambiguities in the denominator.
Punctuate equations with commas or periods when they are part of a sentence:
$$\Gamma_2 a^2 + \Gamma_3 a^3 + \Gamma_4 a^4 + ... = \lambda \Lambda(x),$$
where $\lambda$ is an auxiliary parameter.

Be sure that the symbols in your equation have been defined before the
equation appears or immediately following.
Use ``(1),'' not ``Eq. (1)'' or ``Equation (1),''
except at the beginning of a sentence: ``Equation (1) is ...''.

   \begin{figure}[thpb]
      \centering
      %\includegraphics[scale=1.0]{figurefile}
      \caption{caption}
      \label{figurelabel}
   \end{figure}



\section{Conclusions}

\subsection{Conclusions}





\section{Acknowledgements}

The authors gratefully acknowledge the contribution of reviewers' comments, etc. (if desired). Put sponsor acknowledgments in the unnumbered footnote on the first page.


References are important to the reader; therefore, each citation must be complete and correct. If at all possible, references should be commonly available publications.

\begin{thebibliography}{99}

\bibitem{c1}
J.G.F. Francis, The QR Transformation I, {\it Comput. J.}, vol. 4, 1961, pp 265-271.

\bibitem{c2}
H. Kwakernaak and R. Sivan, {\it Modern Signals and Systems}, Prentice Hall, Englewood Cliffs, NJ; 1991.

\bibitem{c3}
D. Boley and R. Maier, "A Parallel QR Algorithm for the Non-Symmetric Eigenvalue Algorithm", {\it in Third SIAM Conference on Applied Linear Algebra}, Madison, WI, 1988, pp. A20.

\end{thebibliography}

\end{document}
